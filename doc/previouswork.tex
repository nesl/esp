\section{Previous Work}
%IEEE standard

\subsection{Schema Languages}
\subsubsection{SensorML}
The Open Geospatial Consortium (OpenGIS) is developing a standard
called SensorML \cite{opengis:sensorml} which intends to describe a
sensor system in great detail. It uses GML \cite{opengis:gml}, a XML
standard also from OpenGIS, to detail the geospatial properties of
sensors. For example, it includes the relative positions of sensors if
a system has multiple sensors as well as detailed description of the
sensors themselves. It is possible to describe what physical entities
a sensor measures, what accuracy the sensor can achieve, who the
manufacturer is, etc.

Additionally, SensorML includes basic functionalities to describe processes
which handle and transform sensor data. For example, if one has a
small weather station which reports the current wind speed and the air
temperature, one can create a process which uses these two values and
outputs the wind-chill factor by applying the following formula
\[T_c = 35.74 + 0.6215\cdot T - 35.75\cdot V^{0.16} + 0.4275\cdot T
\cdot V^{0.16}\]
where $T_c$ is the wind-chill temperature, $T$ the current air
temperature and $V$ is the wind speed.

SensorML is part of the ``Sensor Web Enablement and OpenGIS
SensorWeb'' area of interest of the OpenGIS consortium. This entity
intends to develop ``interoperability interfaces and meta-data
encodings that enable real time integration of heterogeneous sensor
webs''. Other standards under development in this area are, as
described on the OpenGIS website:
\begin{itemize}
\item Observations \& Measurements - Information model and encodings for
observations and measurements.
\item Sensor Observation Service - Service for managing deployed
sensors and retrieving sensor data.
\item Sensor Planning Service - Service to assist in ``collecting
feasibility plans'' and to process collection requests for a sensor or
group of sensors.
\item Web Notification Service - Service to manage dialogue between a
client and Web service(s) for long duration asynchronous processes.
\end{itemize}

\subsubsection{TinyML}
SensorML provides a sensor-centric approach for sensor coordination
and description. It is also pretty heavy weight and very
complex. TinyML \cite{ota:tmd} addresses these shortcomings and
implements a lightweight version but still following the basic ideas of SensorML.
Furthermore, TinyML is tailored
to embedded sensor networks. The fundamental elements of TinyML are
the sensor, the platform, and the sensor field. A sensor describes a
specific sensor and its properties. A platform represents a physical
system with a processor, an energy source and a radio
device. Additionally, a platform has a collection of sensors. The
sensors measure basic physical phenomena, like temperature, air
pressure etc. At the larger scale, a sensor field is a collection of
platforms and represents a sensor network.

TinyML has capabilities to define virtual devices. These virtual
devices can be generated in an ad-hoc fashion and group together
different platforms or sensors. This allows the ability to easily query 
a subset of platforms and sensors by taking advantage of virtual devices. 
Each element in
TinyML can have a query or set a flag to indicate a response / actuation
of that component. This is a very simple mechanism to interact with a
sensor network.

The advantages of TinyML are that it gives a universal interface to
interact with a sensor network and it is very
lightweight. Unfortunately, TinyML is centered too much on sensor
networks and can not easily be adapted to other sensor systems. Furthermore, it does 
not have any notion of mobility, i.e., for a sensor network
where nodes move around, and there is no rigorous
implementation available.


\subsubsection {IEEE 1451}
The Institute of Electrical and Electronics Engineers (IEEE) developed
a standard called IEEE 1451 \cite{lee2000iss}, a standard to easily network
transducers. The first part, IEEE 1451.2 was already adopted in 1997,
and was closely followed by IEEE 1451.1 in 1999. The standard defines
ways to find out what transducers are on a network. All of this is
done at a very low level, i.e., this would be used on a platform where
the micro-controller wants to know what sensors are attached to its
input ports and buses.


\subsection{Middle-ware Architectures}
\subsubsection{Atlantis Framework}
The Atlantis Framework \cite{arnaudov2005umh} is based on TinyML but
addresses several of its shortcomings. The basic elements are the same, i.e.,
it can describe fields, platforms, and sensors. Additionally, the
Atlantis Framework adds data handling abstractions, and a query field
for more detailed queries. It makes further improvements by
defining a field task object which can handle asynchronous data
retrieval. For this purpose, it adds an additional data broker which
handles the tasks, and specific broker behaviors to describe how to handle 
the task itself. As a nice roundup, the Atlantis
Framework adds data filters and event subscription possibilities.  On the downside,
there is not a standard way to manage the sensor systems since a registry does not
exist.

\subsubsection{TASK}
The ``Tiny Application Sensor Kit'' (TASK) \cite{buonadonna2005tsn}
was designed for use by end-users with minimal sensor network
knowledge. TASK uses TinyDB as a back-end running on nodes and is thus
tailored towards sensor networks running on Mica2 mote
type systems. Additionally, it can handle only one deployment at a time and
needs to be reconfigured to be used for a different deployment, i.e.,
it can not handle multiple deployments simultaneously.

\subsubsection{Others}
F. C. Delicato et al describe in \cite{delicato2003fws} a general
schema to use Web Service technology in a sensor network, i.e., their
main physical components are sensor and sink nodes. The paper fails to
give an implementation of the schema and does not go beyond general web
service concepts.

In \cite{trossen2005bup} D. Trossen and D. Pavel describe a
middle-ware application for smart phones. Their architecture consists
of several different components. They have entities for event
delivery, acquisition, query resolution, aggregation, access control,
storage and even registration and availability. The framework is
mainly tailored towards cellular systems and they claim to have a
test-bed running, though an implementation was not available at the
time of writing this report.

\subsection{Web Interfaces}
The overall architecture of the ESP framework revolves around using web services as
a platform construct.  By using web services, the framework provides support
for interoperable machine to machine interaction.  Specifically, we take advantage of
the communication protocol and service description areas of the web services protocol.
The interface to the registry and also to the actual systems is through the Simple Object
Access Protocol (SOAP) and the actual interfaces are described using the Web Services
Description Language (WSDL) \cite{curbera2002uws}.  ESPml is used to convey the specific
intent of the various calls by providing data necessary to perform the actions 
defined by the interface calls.

Current use of web services is very limited in the sensor network realm.  Instead of providing
access to the actual nodes that are on the system, in terms of sensors, most implementations use
web services as a method to access data that is stored on a database management unit or an aggregation
unit \cite{buonadonna2005tsn},\cite{codeblue}.  The queries to these nodes might be complex and involve 
the query being disseminated throughout the sensor network, but the actual web service interface is still 
typically confined to interacting with a small set of nodes that are at a higher data granularity 
level than the actual sensors in the system. 

Other uses of web services in the sensor network area include 
services that enable sensor data to be published to a particular source that then provides querying services 
on the data that is collected \cite{sensorbase},\cite{senseweb}.  Also, there are numerous applications that
provide web interfaces not based on web services but instead custom interface methods.  Overall, based on the
current landscape of sensor networks and how web services are being used, there is definitely a gap that can
be filled by a unifying system such as the ESP framework that provides multiple services for a sensor network
platform.
 
