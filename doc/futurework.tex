\section{Future Work} \label{sec:future_work}
There are a few different areas that should be investigated further
in terms of this project.  First, some additions need to be made to
the ESPml schema language in order to more accurately describe certain
types of systems.  Also, there are changes that need to be made to the
architecture in order to scale the platform as a whole.  Furthermore, there are
some issues related to privacy, authentication, and standardization
that need to be addressed.  Finally, some additional systems and clients need to
be made for the architecture that can act as services that other
components could use.
\subsection{Standardization}
One of the main actions that needs to be performed in terms of ESPml is
standardization of a few different aspects of the schema.  ESPml
allows different sensor types to be defined.  Currently, these types
can be any arbitrary name.  But there should be a standard naming
convention for the sensor types.  This will enable users that use the
framework to interpret the sensor types programmatically.
Furthermore, certain sensor types should have standard functions that
need to be defined.  These functions will have a certain prototype
associated with them and a standard output as well.  For instance, one
can expect all sensors that observe some type of phenomena and
quantify it in terms of a measurement to have a function to get the
current value.  In addition, they should be able to get the average
over a period of time and to set the sampling rate.  Having
this type of standardization will make describing sensors that are
similar easier since the same schema constructs can be repeated and
guarantee the end user certain basic functionality for most sensor
types.
\subsection{Mobility and Availability}
When describing sensor systems, two attributes that need to be
expressed in some fashion is mobility and availability.
Mobility deals with the notion of whether the system actually moves.  Since the ESPml
schema has a location tag as part of the description language, if the
systems are mobile then this location tag may not be reliable.  To
represent the difference between mobile systems and non-mobile
ones, a mobility tag can be added to different constructs of a system
to indicate that the location might not be reliable based on the information
in the registry.  At this point, the location that is in the actual ESPml
description serves as the last provided location by the sensor system
or the location that the sensor system is actually located at with the
highest probability.  In the next revision of the schema, a confidence
interval can be given to the location indicating how likely the node
will be at that particular location.  Also, if the mobile tag exists, a function
will be required to specify the exact location for that system.  

In terms of availability, one would introduce timestamps and a time to
live counter on the registry.  Basically, when a system is registered
it would be assigned a time to live in which the system is
required to re-contact the registry to verify that it is still
available.  The client program can then check the
registry to get the time to live requirement and at what time the
system last updated the registry to evaluate the availability
confidence level for a system.  Furthermore, statistics can be kept on
the registry about the quality of the system connection.
Overall, both mobility and availability are important aspects of a
sensor system that need to be addressed in the ESPml framework.

\subsection{Authentication and Confidentiality}

There are some key issues that need to be addressed for the ESP framework to
succeed that revolve around security.  Basically, the idea of
having certain services private, users authenticated, and keeping data
confidential are necessary.  There are several methods to address this
issue.  One can imagine creating a system in the actual ESP framework
that enables these attributes to be realized, but a better approach is
to leverage technologies that are already in the market that provide
similar capabilities.

Since ESP framework revolves around the web services platform, a
natural candidate that needs to be analyzed is web service security (WSS
or WS-Security) \cite{securews}.  Basically, WSS is a set of
enhancements to the SOAP messaging protocol in order to provide the
protection needed for authenticity, integrity, and confidentiality of
a message \cite{wss}.  In another words, WSS defines the structure for
SOAP that enables these properties to take place in the actual message
document and for the actual functionality that is provided.  Using XML
Signatures to sign data provides integrity.  XML Signatures contains a
section which has information about the signature algorithm used,
canonicalization method used, the digest value, and key values.  By
checking the signature one can determine if the actual messages
between various sources has been tampered in some fashion.  Confidentiality
is provided by employing encryption techniques under the standards of
XML Encryption. The elements of the SOAP message are
encrypted instead of in plain text.  Finally, authenticity can be
provided by using a key management scheme that can be provided by the
registries.  The registries would provide the public and
private key infrastructure and manage keys for systems and clients
that need to use those systems.  Private keys are kept a secret and
public keys are distributed freely.  The registries could generate the
key pair for each of the different clients that would want to use a
system.  The actual system can then send the message to the registry
to authenticate using a certain method and then a secure connection could
be established between the client and system since there will be a
direct connection.  Also, restrictions related to who and to what capacity
the functions are exposed can be boot strapped through this key management
protocol

In addition to digital signatures, encryption methods, and key management
methods, there also exists policy languages for web services.  This includes
XACML and SAML standards that provide an authorization framework \cite{godik:oea}.  In this
type of model there exists a PEP (Policy Enforcement Point), PDP (Policy
Decision Point), PIP (Policy Information Point), and PAP (Policy Administration
Point).  The PEP takes a request from a client and sends the request to the PDP.
The PDP then gets policies from the PAP and gets attributes using the PIP concerning
the subject, the resource, and the environment and makes a decision for the request.
The request is sent back to the PEP from the PDP and then conveyed to the user.

Clearly, there needs to be more investigation in this space to make a
system that would work correctly for a framework of this type.
Overall, previous work in this area will serve as inspiration in
order to come up with a scheme that will be effective.

\subsection{Selective Sharing and Verification}

Issues related to providing capabilities such as selective sharing and
verification of certain information adds an extra dimension to the ESP
framework.  Selective sharing refers to the idea of providing
different services or outputting various granularity of data
depending on the client that is actually querying for the service.
This can be done on the systems themselves, but as scale gets higher,
one can imagine mediators between the clients and the actual systems
performing this type of action.  Also, in addition to selective
sharing, the concept of verifying certain aspects of the system is an
interesting idea to consider.  For instance, if the framework provided
a method to verify the location of a sensor system and attach a
confidence level associated with this index, then this would be useful
for client applications. One can imagine mediators being involved in
storing or providing this information in some form.

\subsection{Scalability} 

As more sensors are added and query requirements diversify,
scalability becomes an issue with the ESP framework.  Managing a
large amount of sensors with one registry is not feasible without
having performance issues related to queries and management.  Thus, a
hierarchy for registries needs to be developed.  The Domain Name
System (DNS) platform can be used as an example model
\cite{mockapetris:ddn}. When a system is registered, a unique name
could be provided to the system.  An example naming scenario would be to
adopt the unique name based on how domain names are formatted. Thus,
there will be top level domains and then sub domains associated with
sensors that are somehow related.  For instance, sensor systems
located in the same region might share a certain level in the
naming convention.  Then, the actual registries could be arranged in a
tree fashion where each registry is responsible for a certain zone of
entries.  There would be a a set of root registries that handle the
top-level construct in the naming convention.  If the query that comes
cannot be addressed by the top-level registry then the request
would be passed along the hierarchy to a registry that can actually
handle the request.

In terms of queries, location is the only aspect that can be
searched for in terms of finding systems.  But one can imagine other
querying methods.  For instance, the type of sensor, functionality of
sensor systems, and a common naming convention can all be attributes
that can be queried for by clients.  These types of searching
capabilities for sensor systems need to be incorporated into the
registry to make it usable by a larger set of clients.

\subsection{Client Applications and Systems} 

Currently, there are many different types of network systems
that are incorporated into the ESP framework.  But there are not many
clients that use the actual system.  An example Google Map client
interface is provided that enables searching and 
interactivity through a map interface, but one can imagine a larger
plethora of services that can be created on the client side.  Some
necessary clients include a database archiving entity that takes and
stores readings from various sensors that are registered in the ESP
system.  The actual archiver then could add itself as a system
that other clients could actually interact with if needed.  There are
also more sophisticated analysis services that can be provided as
clients by using the capabilities of popular statistical and signal processing 
software components such as R, ESRI, and Matlab.  Finally, we can imagine
using Google Earth or other mapping tools to serve as
further client programs for users to visually query and interact with
various components in the system.

