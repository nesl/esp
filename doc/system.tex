\section{Detailed System Design}
\begin{table*}
  \begin{center}
  \begin{tabular}{|l|llllll|}
    \hline
    \textbf{Name} & \textbf{Fields} & & & & &\\ \hline \hline
    System & id & systemURI & description & xml & & \\
    Field & id & systemId & fieldKey & xml & & \\
    Platform & id & fieldId & platformKey & xml & &\\
    Sensor & id & platformId & sensorKey & xml & &\\
    Location & id & referenceTable & referenceId & xml & & \\
    Polygon & id & locationId & xml & posList & &\\
    Point & id & locationId & xml & latitude & longitude & altitude \\
    \hline
  \end{tabular}
  \end{center}
  \caption{SQL Table Definitions for Registry Entries}
  \label{tab:registrysql} \end{table*} In addition to the ESPml schema language, the framework provides an architecture to
enable registration and interaction with sensor networks.  The basic two components involved in this prototype are the registry
and a system.  The following section details the design of these entities and goes over their interfaces.  

\subsection{Registry} 
The registry is the component that serves as a repository for sensor systems.  It provides services that
enables sensor systems to be added, updated, deleted, and queried for.  In terms of its architecture, one can think of it as a
database back-end with a web service front-end.  The sensor systems are registered by providing an ESPml document.  Thus, the
database structure is modeled similar to the hierarchy of the ESPml schema.  Table \ref{tab:registrysql} shows the definitions
of the various tables that are involved in storing a sensor system in the registry database.  One of the main aspects of the
database structure that needs to be pointed out is how locations, polygons, and points are represented.  Instead of storing them
as pure XML strings, the basic components of each of these structures are represented as fields in the tables.  This enables
efficient searching especially when location based queries are initiated on the registry.  To make the registry faster, a more
thorough design analysis of the database structure needs to be made and is marked for future work.

In terms of interfaces, the registry uses web services via SOAP to implement remote procedure calls.  The following functions are
exposed as part of the service: \verb|register|, \verb|unRegister|, \verb|update|, and \verb|listSystems|.  

\begin{itemize}

\item \textbf{register}

The register function takes an ESPml document that
describes a system and adds it to the database.  If there is already a system in the registry with the same identifier, 
then the old system will be deleted and the new one will be added.  

\item \textbf{unRegister}

In the unRegister case, again a ESPml document will be provided
as input, and the registry simply deletes the system described from the database.  

\item \textbf{update}

The update function is similar to sending the same system
twice via the register command.  The purpose is to update an existing entry.  If the entry does not exist, then it will be added.

\item \textbf{listSystems}

The listSystems function takes a low overhead ESPml document that contains a polygon area that is described as a point string.  The registry responds
with a ESPml document with all the systems, that are part of the polygon area.  

\end{itemize}
If there is a problem due to improper ESPml formatting or
inconsistencies exist in the actual call 
when compared to the database, then an error string is returned.  Standardization of the return value types needs to be made and this is considered an essential next step.

\subsection{System}

The system component represents the actual sensor system that client applications and other sensor systems can access.  Essentially,
the system component is a gateway to the sensor network as a whole.  Interactions with the sensor network occurs through a web service
interface named execute.  

\begin{itemize}

\item \textbf{execute}

Once a consuming program finds the appropriate system they want to interact with, the component runs the
execute command with an ESPml that specifies what functions to run on the system.  The result of the execute command is a lightweight ESPml document
that contains the output of the functions.  The output elements consist of the type attribute and a URI field.  The actual code necessary
to execute the function is defined by the individual sensor network.  Also, the availability of the output is also determined by the
sensor network.  
\end{itemize}
Overall, the framework dictates that the system needs to implement the functions that it specifies, provides a standard method
to actually interact with the system through the execute web service RPC call, and provides a standard way to format the response or output
for the execution.

