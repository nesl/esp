\section{Introduction}
Embedded sensor networks have started becoming useful in a wide variety
of environments, and there have been numerous research efforts that
show how these sensor networks can be applied to such areas ranging
from household situations to scientific pursuits.  For instance,
sensor networks have been used successfully in improving agriculture
procedures in vineyards \cite{brooke:vineyard}, providing better
insight into the learning process in educational institutions
\cite{srivastava:smart}, and detection and classification of objects
in military settings \cite{li:detection}.

As the potential for sensor networks in various fields have been realized, 
software frameworks have been 
developed to make deployments relatively easy to configure, maintain, and use. 
For instance, Tiny Application Sensor Kit (TASK) \cite{buonadonna2005tsn} focuses
on providing a sensor network system in a box, where installation is fairly easy, 
field tools can be used for remote management and health monitoring, and client tools
are available for getting specific information from the system.  Mote-View \cite{turon2005mvs}
addresses the issue of sensor network monitoring in a fine grained fashion 
by providing a suite of tools to visualize the network health of a particular node 
in a system by providing data related to bandwidth, congestion, and throughput.  But most of
these architectures are focused on a single sensor network and are often coupled
with the underlying software, such as the operating system, that is running on the implementation.

Since sensor networks have a large application space where they can be useful and setting
up and monitoring a sensor system is becoming easier, the proliferation of sensor networks has
increased immensely.  But due to the application space variance, there is tremendous heterogeneity
in the logic for interfacing and collecting data on these systems.  The ESP framework provides
the middle-ware architecture to bridge the gap between various sensor systems and enables a standard
way to manage, query, and interact with the various sensor network setups.  

The ESP framework consists of a sensor network description language, in the form of an XML schema (ESPml), that is
used to fully describe a sensor network in terms of location, setup, type of data that is provided, 
and any commands that can be enacted on the system.  These details can be provided
for various granularities ranging from the system as a whole to a specific sensor that is on a particular platform.  
Also, the framework defines an architecture that enables heterogeneous sensor networks
to be registered, queried and interacted with through a common interface available using a web services.
This system architecture, along with the use of ESPml, is demonstrated by registering a variety of sensors and
interacting with them through a Google Map end user interface.

The rest of the paper is organized as follows.  We discuss previous work including schema languages for describing
various attributes in sensor networks that are already available, middle-ware architectures that exist in the sensor
network realm, and existing web interfaces for sensor network systems in Section 2.  A general overview of the system
as a whole is provided in Section 3.  Section 4 contains details about the ESPml XML schema.  A detailed explanation of the ESP framework
registry and system architecture are provided in Section 5.  Work related to using and evaluating the ESP framework is in Section
6.  Section 7 and 8 contain guidelines for future work and the
conclusion for the paper.  Finally, Section 9 contains acknowledgments.  Also, we have an Appendix that contains
details about the code structure of the implementation, extra information about the ESPml schema, and figures related to the example
systems that were used to evaluate the framework.

